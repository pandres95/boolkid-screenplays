\documentclass{screenplay}[2018/01/07]
\usepackage[utf8]{inputenc}
\usepackage{screenplay-custom}

\title{Bool Kid — Series 1, Episode 1: The Professional Procrastinator}
\author{Pablo Andrés Dorado Suárez}

\address{Bogotá, DC \\ 111311 \\
    CO \\
    im@pandres95.me}

\begin{document}
    \coverpage

    \fadein
    BOOL KID is the show host.

    ANTAGONIST is a voice tipically done by Bool Kid, that antagonizes what a
    character says, and can be heared by the characters on scene.

    \vspace{1em}
    \begin{center}
        ESCENA 1
    \end{center}
    \vspace{2em}

    \extslug[day]{A Park}

    BOOL KID is in the center of the camera frame, 1.5m away from it.

    \begin{dialogue}{Bool Kid}
        Hello, world! I'm BoolKid.
    \end{dialogue}

    \begin{dialogue}[off camera]{Antagonist}
        Wow! That was, by far, the most cliché introduction EVER.
    \end{dialogue}

    BOOL KID looks at the left, with a disgruntled face.

    \begin{dialogue}{Bool Kid}
        As if you had a better idea.
    \end{dialogue}

    \begin{dialogue}[off camera]{Antagonist}
        How about this\dots
    \end{dialogue}

    BOOL KID leans at the left side of the scene, then reaches at it. His face
    disappears for a while from the camera, as if reading a script.

    \begin{dialogue}{Bool Kid}
        Do you really want me to say that?
    \end{dialogue}

    \begin{dialogue}{Antagonist}
        Yup!
    \end{dialogue}

    \begin{dialogue}[sighing]{Bool Kid}
        Ok.
    \end{dialogue}

    \transition{Quick cut}

    BOOL KID comes back to initial place.

    \begin{dialogue}{(cont.d)}
        I'm BoolKid, and this is my YouTube channel.
    \end{dialogue}

    \begin{dialogue}[shouts off camera]{Antagonist}
        Boring!
    \end{dialogue}

    BOOL KID expresses anger in face.

    \begin{dialogue}{Bool Kid}
        Now what?!
    \end{dialogue}

    \begin{dialogue}[off camera]{Antagonist}
        I meant\dots\ this one.
    \end{dialogue}

    BOOL KID reaches at right side of scene. Leans at right and acts as if
    reading something.

    \begin{dialogue}{Bool Kid}
        Ah! I see.
    \end{dialogue}

    \transition{Quick cut}

    BOOL KID comes back to initial place.

    \begin{dialogue}{(cont.d)}
        Hey everyone! I'm Bool Kid and welcome to Coder Diaries.
    \end{dialogue}

    ANTAGONIST booes, off camera.
    BOOL KID looks at left.

    \begin{dialogue}[shouting]{Bool Kid}
        You know what? Go f**k yourself.
    \end{dialogue}

    BOOL KID comes back to initial place.

    \begin{dialogue}{Bool Kid}
        I'm Bool Kid and I have some bad news.

        Now, let's talk about\dots
    \end{dialogue}

    \transition{Cut to}

    \extslug[night]{The same park}

    \begin{dialogue}[slows speech]{(cont.d)}
        \dots procrastination.
    \end{dialogue}
    \fadeout

    \theend
\end{document}
