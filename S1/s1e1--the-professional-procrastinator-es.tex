\documentclass{screenplay}[2018/01/07]
\usepackage[utf8]{inputenc}
\usepackage[T1]{fontenc}
\usepackage{../Common/screenplay-custom}
\usepackage{natbib,hyperref}

\title{Bool Kid | Temporada 1, Episodio 1: El Procrastinador Profesional}
\author{Pablo Andrés Dorado Suárez}

\address{Bogotá, DC \\ 111311 \\
    CO \\
    im@pandres95.me}

\begin{document}
    \coverpage

    \fadein
    BOOL KID es el presentador del programa.

    ANTAGONISTA es una voz típicamente hecha por BOOL KID, que antagoniza lo que
    un personaje dice, y puede ser escuchada por los personajes en escena.

    \scenetitle{Escena 1}

    \extslug[día]{Un parque}

    BOOL KID está en el centro del encuadre de la cámara, a 1.5m de distancia de
    esta.

    \begin{dialogue}{Bool Kid}
        Hola, Mundo! Soy Bool Kid.
    \end{dialogue}

    \begin{dialogue}[off camera]{Antagonista}
        Wow! Eso fue, de lejos, la introducción más cliché de TODOS LOS TIEMPOS.
    \end{dialogue}

    BOOL KID mira a la izquierda, con una cara de disgusto.

    \begin{dialogue}{Bool Kid}
        Como si tuvieras una mejor idea.
    \end{dialogue}

    \begin{dialogue}[off camera]{Antagonista}
        Que tal esto\dots
    \end{dialogue}

    BOOL KID se inclina al lado izquierdo de la escena, luego se acerca a este.
    Su cara desaparece por un momento de la cámra, como si estuviera leyendo un
    libreto.

    \begin{dialogue}{Bool Kid}
        ¿De veras quieres que diga eso?
    \end{dialogue}

    \begin{dialogue}{Antagonist}
        Sip!
    \end{dialogue}

    \begin{dialogue}[sighing]{Bool Kid}
        Ok.
    \end{dialogue}

    \transition{Quick cut}

    BOOL KID vuelve al lugar inicial.

    \begin{dialogue}{(cont.d)}
        Soy Bool Kid y este es mi canal de YouTube.
    \end{dialogue}

    \begin{dialogue}[shouts off camera]{Antagonista}
        ¡Aburrido!
    \end{dialogue}

    BOOL KID expresa enojo en su rostro.

    \begin{dialogue}{Bool Kid}
        ¡¿Y ahora qué?!
    \end{dialogue}

    \begin{dialogue}[off camera]{Antagonista}
        Me refería\dots\ a esto.
    \end{dialogue}

    BOOL KID se acerca al lado derecho de la cámara. Se inclína hacia la derecha
    y actúa como si estuviera leyendo algo.

    \begin{dialogue}{Bool Kid}
        ¡Ah! Ya veo.
    \end{dialogue}

    \transition{Quick cut}

    BOOL KID vuelve al lugar inicial.

    \begin{dialogue}{(cont.d)}
        ¡Hola a todos! Soy Bool Kid y bienvenidos al diario del programador.
    \end{dialogue}

    El ANTAGONISTA abuchea, fuera de cámara.
    BOOL KID mira a la izquierda.

    \begin{dialogue}[shouting]{Bool Kid}
        ¿Sabes qué? J****e.
    \end{dialogue}

    BOOL KID vuelve al lugar inicial.

    \begin{dialogue}{Bool Kid}
        Soy Bool Kid y les tengo malas noticias.

        Ahora, hablemos de la\dots
    \end{dialogue}

    \transition{Cut to}

    \extslug[night]{The same park}

    \begin{dialogue}[slows speech, shows three-day beard]{(cont.d)}
        \dots procrastinación.

        Lo sé, lo sé, dejé esta parte para más tarde. Tres\dots\ días más tarde.
        Y aunque pienso que este pequeño chiste sería suficiente, sólo en caso
        de que no lo hayas entendido, permíteme explicarlo.
    \end{dialogue}

    \scenetitle{Scene 2}

    \begin{dialogue}[off camera]{(cont.d)}
        Como lo explican los psicólogos autralianos Gery Beswick —si, Gery con
        e— y Leon Mann, y más tarde los investigadores suecos Alexander Rozental
        y Per Carlbring:
        \begin{quote}
            La procrastinación es una falla autorregulatoria omnipresente que
            afecta a aproximadamente un quinto de la población adulta y a la
            mitad de la población estudiantil. Se define como la demora
            voluntaria de un curso de acción previsto a pesar de estar peor
            como resultado de este retraso.
        \end{quote}\citet{rozental2014}
    \end{dialogue}
    \fadeout

    \theend

    \pagebreak
    \bibliographystyle{plainnat}
    \bibliography{s1e1references}
\end{document}
