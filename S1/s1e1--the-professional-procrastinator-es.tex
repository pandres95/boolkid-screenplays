\documentclass{screenplay}[2018/01/07]
\usepackage[utf8]{inputenc}

\title{Bool Kid — Temporada 1, Episodio 1: El Procrastinador Profesional}
\author{Pablo Andrés Dorado Suárez}

\address{Bogotá, DC \\ 111311 \\
    CO \\
    im@pandres95.me}

\begin{document}
    \coverpage

    \fadein
    BOOL KID es el presentador del programa.

    ANTAGONISTA es una voz típicamente hecha por BOOL KID, que antagoniza lo que
    un personaje dice, y puede ser escuchada por los personajes en escena.

    \vspace{1em}
    \begin{center}
        ESCENA 1
    \end{center}
    \vspace{2em}

    \extslug[día]{Un parque}

    BOOL KID está en el centro del encuadre de la cámara, a 1.5m de distancia de
    esta.

    \begin{dialogue}{Bool Kid}
        Hola, Mundo! Soy Bool Kid.
    \end{dialogue}

    \begin{dialogue}[off camera]{Antagonista}
        Wow! Eso fue, de lejos, la introducción más cliché de TODOS LOS TIEMPOS.
    \end{dialogue}

    BOOL KID mira a la izquierda, con una cara de disgusto.

    \begin{dialogue}{Bool Kid}
        Como si tuvieras una mejor idea.
    \end{dialogue}

    \begin{dialogue}[off camera]{Antagonista}
        Que tal esto\dots
    \end{dialogue}

    BOOL KID se inclina al lado izquierdo de la escena, luego se acerca a este.
    Su cara desaparece por un momento de la cámra, como si estuviera leyendo un
    libreto.

    \begin{dialogue}{Bool Kid}
        ¿De veras quieres que diga eso?
    \end{dialogue}

    \begin{dialogue}{Antagonist}
        Sip!
    \end{dialogue}

    \begin{dialogue}[sighing]{Bool Kid}
        Ok.
    \end{dialogue}

    \begin{flushright}
        QUICK CUT:
    \end{flushright}

    BOOL KID vuelve al lugar inicial.

    \begin{dialogue}{(cont.d)}
        Soy Bool Kid y este es mi canal de YouTube.
    \end{dialogue}

    \begin{dialogue}[shouts off camera]{Antagonista}
        ¡Aburrido!
    \end{dialogue}

    BOOL KID expresa enojo en su rostro.

    \begin{dialogue}{Bool Kid}
        ¡¿Y ahora qué?!
    \end{dialogue}

    \begin{dialogue}[off camera]{Antagonista}
        Me refería\dots\ a esto.
    \end{dialogue}

    BOOL KID se acerca al lado derecho de la cámara. Se inclína hacia la derecha
    y actúa como si estuviera leyendo algo.

    \begin{dialogue}{Bool Kid}
        ¡Ah! Ya veo.
    \end{dialogue}

    \begin{flushright}
        QUICK CUT:
    \end{flushright}

    BOOL KID vuelve al lugar inicial.

    \begin{dialogue}{(cont.d)}
        ¡Hola a todos! Soy Bool Kid y bienvenidos al diario del programador.
    \end{dialogue}

    El ANTAGONISTA abuchea, fuera de cámara.
    BOOL KID mira a la izquierda.

    \begin{dialogue}[shouting]{Bool Kid}
        ¿Sabes qué? J****e.
    \end{dialogue}

    BOOL KID vuelve al lugar inicial.

    \begin{dialogue}{Bool Kid}
        Soy Bool Kid y les tengo malas noticias.

        Ahora, hablemos de la\dots
    \end{dialogue}

    \begin{flushright}
        CUT TO:
    \end{flushright}

    \extslug[night]{The same park}

    \begin{dialogue}[slows speech]{(cont.d)}
        \dots procrastinación.
    \end{dialogue}
    \fadeout

    \theend
\end{document}
